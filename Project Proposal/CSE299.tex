\documentclass[12pt]{article}
\usepackage{textcomp}
\usepackage[utf8]{inputenc}
\usepackage[english]{babel}
\usepackage{gensymb}
\usepackage{float}
\usepackage{hyperref}
\hypersetup{
    colorlinks=true,
    linkcolor=blue,
    filecolor=magenta,      
    urlcolor=blue,
}
 
\urlstyle{same}
\usepackage{graphicx}
\graphicspath{ {./images/} }
\title{Farsad's Store}
\begin{document}
\begin{titlepage}
    \begin{center}
        \includegraphics[width=0.4\textwidth]{NSU.png} \\
        Department of Electrical \& Computer Engineering\\
        {\scshape\LARGE North South University \par}
        \today
        \vspace{1cm}
        
        {\scshape\Large Farsad's Store\par}
 
        \vspace{0.5cm}
        {\scshape\Large CSE299: Junior Project Design\par}
        Section 2 \\ Fall 2020
             
        \vspace{0.5cm}
        Supervised by\par
	    Shaikh Shawon Arefin Shimon (SAS3) \par
        \vspace{0.5cm}
        \textbf{Team Members (Group 02)} \par
        \begin{tabular}{c c c}
            Name & ID & Email \\
            ILMIAT FARHANA & 1712428042 & ilmia.farhana@northsouth.edu \\
            Sakib Sadman Shajib & 1731201042 & sakib.shajib173@northsouth.edu \\
        \end{tabular}

        \vfill
             
    \end{center}
\end{titlepage}
\section{Introduction}
The proposed application will be selling products such as computers, laptops, keyboard, house, headphones, SSDs and others related technology gadgets. E-commerce or Online Store is fast gaining ground as an accepted and used business paradigm. More and more business houses are implementing web sites providing functionality for performing commercial transactions over the web. It is reasonable to say that the process of shopping on the web is becoming a commonplace. \par
The objective of this project is to develop a product catalog where products can be viewed and users can buy items online using their PC or smartphone. \par
This is a virtual store where customers can browse the catalog and add products they want to buy to the cart. Then they can checkout the items in the cart and provide to pay online or Payment on Delivery.

\section{Description}
The general objective of the project is to design and develop an online shopping experience. The system should the following features: \par
\begin{itemize}
    \item A catalog of products with multiple categories.
    \item User based cart management.
    \item Online Payment or Payment on Delivery.
    \item Delivery Management System.
    \item Inventory Management System.
    \item Accounting Management System.
\end{itemize}
There will be three types of user: \par
\begin{itemize}
    \item Admin
    \item Employee
    \item Customer
\end{itemize}

\section{Features}
\subsection{Catalog of Products}
Like shelves on a physical store, our online store will have products which will be divided into different categories, the categories will be dynamic, so you can add new categories without accessing the code.

\subsection{User Cart}
Every time an user adds a product to their cart, it'll stay there till the user wants checkouts it out or cancels it.

\subsction{Online Payment}
There will be an online payment gateway for our users to pay using their card or digital wallet, there will be a payment on delivery option as well.

\subsction{Delivery}
Users will be able to track their orders in the phone and see status of the delivery.

\subsction{Inventory}
The system will automatically keep track of product availability depending on the data provided by an employee.

\subsction{Accounting}
The system will keep track of all the transactions of the system; keep track of purchases, invoices, payments etc.

\section{Technology}
We decided to use Django with REST API as our backend and React for our Frontend.

\subsection{Frontend}
Since our site needs to be attractive and smooth, we planned to use React.JS, an web framework of JavaScript which is developed by Facebook. This will give us plenty of wiggle room on how much customization we can make to the system. The Frontend will only deal with data viewing, beautification of site, and send data back and from to the API.

\subsection{Backend}
We are using Django, a python web framework as our backend system and PostgreSQL as our Relational Database System. Django gives us a lot of room to add functionality to our system. We will be building an API, using the Django REST framework, which will bridge the connection between backend and frontend. \par
Django will take care of authentication. We plan to use third part authentication API to make sure people can login with their Google Account/ Facebook or other system.

\subsubsection{Payment method}
We are planning to use SSLCommerz's Sandbox as our payment gateway.

\section{DevOps}
The code will be configured with CI and CD using GitHub Actions. It'll automatically create pull requests when some basic tests are passed by our code, and after we verify the pull requests to the main branch, the CD will automatically deploy it to our cloud provider. \par
We will be using GCP/AWS/Azure as our Cloud provider. We will use their Kubernetes solutions to automatically orchestrate our dockerized application, so it runs and scales automatically on the fly.

\end{document}